\documentclass{article}
\usepackage[utf8]{inputenc}
\usepackage{url}

\title{Australian Athletes Data}
\author{Yehia Abdelmohsen (900193174)}
\date{April 2022}

\begin{document}

\maketitle

\section{About the Data}
The data presented contains information related to Australian athletes. 
\newline
\par
\noindent
The data is obtained from R, its documentation is linked below.
\newline
\par
\noindent
Source: \url{https://vincentarelbundock.github.io/Rdatasets/doc/DAAG/ais.html}


\section{Variables}
\begin{itemize}
    \item Rcc - Red blood cell count. This is a quantitative variable.
    \item Wcc - White blood cell count, per liter. This is a quantitative variable.
    \item Hematocrit - Percent of hematocrit. Hematocrit: The ratio of the volume of red blood cells to the total volume of blood. This is a quantitative variable.
    \item Hg - Hemoglobin concentration in g per decaliter (10 liters). Hemoglobin: A protein that carries oxygen to organs. This is a quantitative variable.
    \item Ferr - Plasma ferritins in ng. Measures the amount of ferritin in blood. Ferritin is a blood protein that contains iron. This is a quantitative variable.
    \item Bmi - Body mass index, in kg/\(m^2\). Body mass divided by the square of the height \(m^2\). This is a quantitative variable.
    \item Ssf - Sum of skin folds. Estimates the percentage of body fat by measuring skin fold thickness. This is a quantitative variable.
    \item PcBfat - Percentage of body fat. This is a quantitative variable.
    \item Lbm - Lean body mass in kg. Total body weight minus body fat weight. This is a quantitative variable.
    \item Ht - Height in cm. This is a quantitative variable.
    \item Wt - Weight in kg. This is a quantitative variable.
    \item Sex - A factor representing the sex of the athlete: female and male. This is a qualitative categorical variable.
    \item Sport - A factor representing the sport the athlete plays B Ball, Field, Gym, Netball, Row, Swim, T 400m, T Sprnt, Tennis, and W Polo. This is a qualitative categorical variable and will be used as the class variable.
\end{itemize}



\end{document}